\documentclass[10pt,landscape,letterpaper]{article}
\usepackage{multicol}
\usepackage[landscape]{geometry}
\usepackage{amsmath,amsthm,amsfonts,amssymb}
\usepackage{mathtools}
\usepackage{tabularx}
\usepackage{fancyhdr}
\usepackage{enumitem}
\usepackage{tabularx}

%
% Styling inspired by tex.stackexchange user "Dror" in this posting:
% https://tex.stackexchange.com/questions/8827/preparing-cheat-sheets
%

\thispagestyle{fancy}
\cfoot{\footnotesize Copyright \textcopyright \ 2020 Brian Schubert}
\rfoot{\footnotesize Revised 2020-04-21}
\lhead{MATH 4545\\Fourier Series and PDEs}
\rhead{Spring 2020 \\ Prof. Solomon Jekel}
%\chead{\large \underline{Midterm Exam ``Cheat Sheet"}}
\renewcommand{\headrulewidth}{0pt}
\setlength{\voffset}{-4pt}

\geometry{top=0.7in,left=.5in,right=.5in,bottom=.5in}


% Redefine section commands to use less space
\makeatletter
\renewcommand{\section}{\@startsection{section}{1}{0mm}%
                                {-1ex plus -.5ex minus -.2ex}%
                                {0.5ex plus .2ex}%x
                                {\normalfont\large\bfseries}}
\renewcommand{\subsection}{\@startsection{subsection}{2}{0mm}%
                                {-1explus -.5ex minus -.2ex}%
                                {0.5ex plus .2ex}%
                                {\normalfont\normalsize\bfseries}}
% Custom squishing for \paragraph
\renewcommand\paragraph{\@startsection{paragraph}{4}{\z@}%
    {0.5ex \@plus1ex \@minus.2ex}%
    {-1em}%
    {\normalfont\small\bfseries}}

% Don't print section numbers
\setcounter{secnumdepth}{0}


\setlength{\parindent}{0pt}
\setlength{\parskip}{0pt plus 0.5ex}

% Notation Definitions
\newcommand{\dd}[1]{\mathrm{d}#1}
\newcommand{\diffop}[2][]{\frac{\dd#1}{\dd #2}}
\newcommand{\matr}[1]{\mathbf{#1}}
\newcommand{\bvec}[1]{\mathbf{#1}}
\newcommand{\vecspace}[1]{\mathbb{#1}}
\newcommand{\transp}{^\mathrm{T}}

\newcommand\cheatsheetmargin{0.2cm}

% -----------------------------------------------------------------------

\begin{document}
\raggedright
\footnotesize
\begin{multicols}{3}


% multicol parameters
% These lengths are set only within the two main columns
%\setlength{\columnseprule}{0.25pt}
\setlength{\premulticols}{1pt}
\setlength{\postmulticols}{1pt}
\setlength{\multicolsep}{1pt}
\setlength{\columnsep}{2pt}

\begin{center}
     \Large{\underline{Preliminaries}} \\
\end{center}

\subsection{Basic Trig Identities}
Let $n \in \mathbb{Z}^+_0$ and $k=2n+1$
\begin{align*}
\sin n \pi &= 0      &   \sin \frac{k\pi}{2} &= (-1)^n & & \cosh x= \frac{e^x + e^{-x}}{2} \\
\cos n \pi &= (-1)^n &   \cos \frac{k\pi}{2} &= 0      & & \sinh x= \frac{e^x - e^{-x}}{2}\\
\cosh 0 & = 1        & \sinh 0               &= 0      & & e^{x}  \cosh (x) + \sinh (x)  
\end{align*}

\subsection{Second Order Constant Coefficient Homogeneous ODEs}

\begin{description}[style=unboxed,leftmargin=\cheatsheetmargin]
    \item[Form:] \qquad\quad $ay'' + by' + cy = 0, \quad a,b,c \in \mathbb{R}$
    \item[Solved by:] $y(x) = C_1 e^{r_1 x} + C_2 e^{r_2 x}, 
    \quad ar^2 + br + c = 0$
    
    {\setlength{\tabcolsep}{2em}
    \begin{center}
    \begin{tabular}{c|c}
        \textbf{Case} & \textbf{Solution} \\
        $b^2 - 4ac > 0$ & $y(x) = C_1 e^{r_1 x} + C_2 e^{r_2 x}$\\
        \hline
        $b^2 - 4ac =0$ &  $\begin{array}{c}r_1 = r_2 = \frac{-b}{2a} \\[\smallskipamount]  y(x) = C_1 e^{r_1 x} + C_2 \mathbf{x}e^{r_1 x}\end{array}$ \\
        \hline
        $b^2 - 4ac < 0$ & $\begin{array}{c}r_1, r_2 = p \pm qi \quad \ \text{where}\\[\smallskipamount]
         p = \frac{-b}{2a},\quad q = \frac{\sqrt{4ac - b^2 }}{2a} \\[\medskipamount]  
        y(x) = C_1 e^{r_1 x} + C_2 \mathbf{x}e^{r_1 x}\end{array}$
    \end{tabular}
    \end{center}
    }
\end{description}

\subsection{Orthogonality Relationships}
For $n,m \in \mathbb{Z}^+$ and $L>0$,
\begin{gather*}
     \int_{-L}^{L} \sin^2 \left(\frac{n\pi x}{L}\right)\, \mathrm{d} x 
     = \int_{-L}^{L} \cos^2 \left(\frac{n\pi x}{L}\right)\, \mathrm{d} x = \begin{cases}
         0, & n \neq m\\
         L, & n = m
     \end{cases}\\
      \int_{-L}^{L} \sin \left(\frac{n\pi x}{L}\right) \cos \left(\frac{n\pi x}{L}\right) \, \mathrm{d} x = 0.
\end{gather*}


\begin{center}
    \Large{\underline{Fourier Series}} \\
\end{center}
Given $f$ defined on a symmetric interval $x\in[-L, L]$, the Full Fourier Series of $f$ is given by
\begin{equation*}\boxed{
f(x) \sim \frac{a_0}{2} + \sum_{n=1}^{\infty} a_n \cos \left(\frac{n\pi x}{L}\right) + \sum_{n=1}^{\infty} b_n \sin \left(\frac{n\pi x}{L}\right)}
\end{equation*}
where
\begin{gather*}
a_0 = \frac{1}{L}\int_{-L}^{L} f(x) \, \mathrm{d}x, \qquad 
a_n = \frac{1}{L}\int_{-L}^{L} f(x) \cos \left(\frac{n\pi x}{L}\right)\, \mathrm{d}x \\
b_n = \frac{1}{L}\int_{-L}^{L} f(x) \sin \left(\frac{n\pi x}{L}\right)\, \mathrm{d}x
\end{gather*}

\paragraph{Convergence Theorem for Fourier Series}
Let $f$ be piece-wise differentiable on $[-L, L]$. The F.S. of $f$ coverges to the periodic extension of $f$ whenever $f$ is continuous and to the average $\frac{1}{2} \left(f(a^-) + f(a^+)\right)$ in general.



\subsection{Fourier Sine and Cosine Series}
If $f$ defined on $x\in[-l, l]$ is $\begin{array}{c}\text{even}\\\text{odd}\end{array}$, then $\begin{array}{c}b_n = 0\\ a_0 = a_n = 0\end{array}$.

Suppose $f$ is defined on $x\in [0, L]$.
\begin{description}[style=unboxed,leftmargin=0pt]
    \item[F.C.S] \begin{gather*}
    f(x) \sim \frac{a_0}2 + \sum_{n=1}^\infty a_n \cos \left(\frac{n\pi x}{L}\right) \quad 
    a_n = \frac{2}{L} \int_0^L f(x) \cos \left(\frac{n\pi x}{L}\right) \, \mathrm dx
    \end{gather*}
     \item[F.S.S] \begin{gather*}
    f(x) \sim \sum_{n=1}^\infty b_n \sin \left(\frac{n\pi x}{L}\right) \ \qquad 
    b_n = \frac{2}{L} \int_0^L f(x) \sin \left(\frac{n\pi x}{l}\right) \, \mathrm dx
    \end{gather*}
\end{description}



\begin{center}
    \Large{\underline{Sturm-Liouville Boundary}}
    \Large{\underline{Value Problems}}
\end{center}


\paragraph{Regular SL-BVP:}
\begin{gather*}
(pf')' + qf + \lambda \sigma f=0   \qquad  f=f(x) \qquad   x \in[a, b]  \tag{1}\\
K_1 f(a) = K_2 f'(a) = 0    \qquad       K_3 f(b) + K_4 f'(b) = 0      \tag{2}
\end{gather*}
where $p,q,\sigma$ are continuous on $(a,b)$, $p'$ is continuous, and $p,\sigma$ are positive

\paragraph{Basic SL-BVP:} $p=1, q=0, \sigma = 1 \implies \boxed{f'' + \lambda f = 0}$
\paragraph{Eigenvalues \& Eignevectors} The value(s) $\lambda$ for which the equation (1) with the boundary conditions (2) has non-zero solutions ar called the eigenvalues of the given problem. The corresponding functions are called eigenfunctions.

\subsection{Four Basic SL-BVP}
\begin{tabular}{ccll}
    \textbf{Boundary Cond.} & \textbf{Eigenvalues} & \textbf{Eigenfunc.} & $n\in$\\
    $f(0) = f(L) = 0$ & $n^2\pi^2/L^2$ & $\sin \left(n\pi x/L\right)$ & $\mathbb{Z}^+$ \\
    $f'(0) = f'(L) = 0$ & $n^2\pi^2/L^2$ & $\cos \left(n\pi x/L\right)$& $\mathbb{Z}^+_0$ \\
    $f(0) = f'(L) = 0$ & $\left[\frac{(2n-1)\pi}{2L}\right]^2$ & $ \sin \left(\frac{(2n-1)\pi x}{2L}\right)$& $\mathbb{Z}^+$ \\
    $f'(0) = f(L) = 0$ & $\left[\frac{(2n-1)\pi}{2L}\right]^2$ & $ \cos \left(\frac{(2n-1)\pi x}{2L}\right)$& $\mathbb{Z}^+$ \\
\end{tabular}

\textbf{Note:} For all Four Basic SL-BVPs, $\int_0^L X_n^2(x) \,\mathrm{d}x = \frac{1}{2}$.

\subsection{Basic SL Theory}
\begin{enumerate}
    \item There are always an infinite number of eignenvalues to (1) and (2) where $\lambda_1 < \lambda_2 < \cdots$ and $\lambda_i \to \infty$.
    
    \item The eigenfunctions corresponding to distinct eigenvalues are always orthogonal with respect to $\sigma$: \begin{equation*}
        \int_a^b f_n(x) f_m(x) \sigma(x)\, \mathrm{d}x = 0 \quad \text{for } n \neq m
    \end{equation*}
    
    \item Every piecewise differenciable function on $[a,b]$ can be written as a sum of a complete set of eigenfunctions: \begin{equation*}\boxed{
        g(x) \sim \sum_{n=1}^{\infty} a_n f_n (x) \quad\text{where}\quad a_n = \frac{\displaystyle \int_a^b g(x) f_n(x)\sigma(x)\, \mathrm dx}{\displaystyle\int_a^b f^2_n(x) \, \mathrm dx}}
    \end{equation*}
\end{enumerate}

\columnbreak

\begin{center}
    \Large{\underline{Separation of Variables}}
\end{center}
\section{The Heat Equation \textnormal{\small ``Diffusion'' of Energy}}
{\normalsize
\begin{gather*}
\mu_t = k\mu_{xx} \quad \mu = \mu(x,t), \quad t>0\\
x\in [0, L], \quad \mu(0,t) = \mu (L, t) = 0
\end{gather*}}
\paragraph{Separation of Variables Hypothesis} We assume
\begin{gather*}
    \mu(x,t) = X(x)T(t) \implies \frac{T'}{kT} = \frac{X''}{X} = - \lambda
    \shortintertext{so}
 T' \sim e^{-\lambda kt}, \quad X'' + \lambda X = 0
\end{gather*}
\paragraph{Fourier Formal Solution}
\begin{equation*}
\mu(x,t) = \sum_{n=1}^\infty c_n  e^{-\lambda kt} X_n(x)
\end{equation*}

\paragraph{Imposing Initial Conditions $\mu(x, 0) = f(x)$}
\begin{equation*}
    f(x) = \sum_{n=1}^\infty c_n X_n(x) \implies c_n =\frac{\displaystyle \int_0^L f(x) X_n(x)\, \mathrm dx}{\displaystyle\int_0^L X_n^2(x) \, \mathrm dx}
\end{equation*}

\section{The Wave Equation  \textnormal{\small ``Propagation'' of Energy}}
{\normalsize
    \begin{gather*}
    \mu_{tt} = c^2\mu_{xx} \quad \mu = \mu(x,t), \quad t>0\\
    x\in [0, L], \quad \mu(0,t) = \mu (L, t) = 0, \quad\mu_t(x, 0) = g(x)
    \end{gather*}}
\paragraph{Separation of Variables Hypothesis} We assume
\begin{gather*}
\mu(x,t) = X(x)T(t) \implies \frac{T''}{c^2 T} = \frac{X''}{X} = - \lambda
\shortintertext{so}
X'' + \lambda X = 0, \quad T'' +\lambda c^2 T = 0
\end{gather*}

\paragraph{Formal Solution}
\begin{equation*}
    \mu(x,t) = \sum_{n=1}^\infty \left( b_{1n} \cos \left(\sqrt{\lambda c} t\right) + b_{2n} \sin\left(\sqrt{\lambda c}t\right) \right)X_n(x)
\end{equation*}

\paragraph{Computing the Coefficients}
\begin{gather*}
 \mu(x,0) = f(x) = \sum_{n=1}^\infty b_{1n} X_n(x) \implies b_{1n} =\frac{\displaystyle \int_0^L f(x) X_n(x)\, \mathrm dx}{\displaystyle\int_0^L X_n^2(x) \, \mathrm dx}\\
  \mu_t(x,0) = g(x) = \sum_{n=1}^\infty \sqrt{\lambda c}b_{2n} X_n(x)  \Rightarrow \sqrt{\lambda c}b_{2n} =\frac{\displaystyle \int_0^L g(x) X_n(x)\, \mathrm dx}{\displaystyle\int_0^L X_n^2(x) \, \mathrm dx}
\end{gather*}

\paragraph{d'Alembert's Solution}
Assume $\mu_t(x, 0) = 0$, i.e. the wave starts from rest.
\begin{equation*}
\mu(x,t) = \frac{1}{2} F(x + ct) + \frac{1}{2} F(x-ct)
\end{equation*}
where $F$ is the odd periodic extension of the initial condition $f$. I.e., to get $\mu(x,t)$, take the initial $f(x)$, translate the odd periodic extension $ct$ units to the left, $ct$ units to the right, and average on $[0, L]$.

\section{Laplace's Equation $\mu_{t} = k(\mu_{xx} + \mu_{yy})$}
We assume $\mu_t = 0$. The resulting solution is called steady state.
Assuming $\mu = XY$, \begin{gather*}
X''Y + Y''X = 0 \implies \frac{X''}{X} = \frac{-Y''}{Y} = -\lambda
\shortintertext{so}
X'' + \lambda X = 0, \qquad Y'' - \lambda Y = 0
\end{gather*}


\vspace{2cm}

.

\vspace{2cm}

.

\vspace{2cm}

.

\vspace{2cm}

.

\vspace{2cm}

.

\vspace{2cm}

.

\vspace{2cm}

.

\vspace{2cm}

.

%text
%
%text
%
%text
%
%text
%
%text
%
%text
%
%text
%
%text
%
%text
%
%text
%
%text
%
%text
%
%text
%
%text
%
%text
%
%text
\end{multicols}
\end{document}