\documentclass[a4paper,12pt]{article} % Copyright (c) 2020  Brian Schubert
\usepackage{amsmath,amsthm,amsfonts,amssymb}
\usepackage{titling}
\usepackage{enumitem}
\usepackage{geometry}
\usepackage[makeroom]{cancel}

\geometry{left=0.75in,right=0.75in,top=1in,bottom=1in}

\theoremstyle{plain}
\newtheorem{problem}{Problem}

% Hide QED symbol
%\renewcommand{\qedsymbol}{}

\begin{document}
% Title
\begin{center}
    \huge
    Math 4545 Take Home \#1

    \vspace{0.5cm}
    \large
    Brian Schubert

    \vspace{0.5cm}
    Due: 02/21/2020
\end{center}

\begin{problem}
    In class used the F.S.S of $f(x)=1$, $0 \leq x \leq \pi$,
    to show that \begin{equation*}
     1-\frac{1}{3}+\frac{1}{5}-\frac{1}{7}+\cdots = \frac{\pi}{4}.
     \end{equation*} 
     Use the F.C.S of $f(x)=x$, $0\leq x \leq \pi$, to find \begin{equation*}
        1 + \frac{1}{2^2} + \frac{1}{3^2} + \frac{1}{4^2} + \cdots
     \end{equation*}
\end{problem}

\begin{proof}[\textbf{Solution}]
    The F.C.S of $f$ is \begin{equation*}
        f(x) \sim \frac{a_0}2 + \sum_{n=1}^\infty a_n \cos \left(nx\right),
        \quad 
        a_0 = \frac{2}{\pi} \int_0^\pi f(x) \, \mathrm dx, \quad
        a_n = \frac{2}{\pi} \int_0^\pi f(x) \cos \left(nx\right) \, \mathrm dx
    \end{equation*}
    
    Computing the coefficients,
    \begin{align*}
        a_0 &= \frac{2}{\pi} \int_0^\pi x \, \mathrm dx = \frac{2}{\pi} \left[ \frac{1}{2}x^2\right]_0^\pi = \pi,\\
        a_n &= \frac{2}{\pi} \int_0^\pi x \cos \left(nx\right) \, \mathrm dx
            = \frac{2}{\pi}\left[\cancelto{0}{\left.\frac{x}{n}\sin(nx)\right|_0^\pi} - \frac{1}{n}\int_0^\pi \sin(nx) \, \mathrm{dx}\right]
            = \frac{2}{n^2 \pi}\Big[\cos(nx)\Big]_0^\pi \\
           &=\frac{2}{n^2 \pi}\Big[(-1)^n - 1\Big]
    \end{align*}
    Therefore, 
    \begin{equation*}
        f(x) = x \sim  \frac{\pi}{2} + \sum_{n=1}^\infty \left(\frac{2}{n^2 \pi}\Big[(-1)^n - 1\Big]\right)\cos(nx)
    \end{equation*}
    
    Since this series converges to the even piecewise extension of $f$, the series evaluated at $x=0$ will equal $f(0)$:
    \begin{align*}
    &&f(0) = 0&= \frac{\pi}{2} + \sum_{n=1}^\infty \left(\frac{2}{n^2 \pi}\Big[(-1)^n - 1\Big]\right) \cdot 1 \\
     &\iff& -\frac{\pi}{2}&=\sum_{n=1}^\infty \left(\frac{2}{n^2 \pi}\Big[(-1)^{n}-1\Big]\right) = \frac{-4}{\pi} + \frac{-4}{3^2\pi} + \frac{-4}{5^2 \pi} + \cdots\\
    &\iff& \frac{\pi}{2}&=\sum_{n=1}^\infty \left(\frac{2}{n^2 \pi}\Big[1-(-1)^{n+1}\Big]\right) = \frac{4}{\pi} + \frac{4}{3^2\pi} + \frac{4}{5^2 \pi} + \cdots\\
    &\iff& \frac{\pi}{4}\cdot\frac{\pi}{2}&=\sum_{n=1}^\infty \left(\frac{2}{n^2 \pi}\Big[1-(-1)^{n+1}\Big]\right) = \frac{\pi}{4}\left(\frac{4}{\pi} + \frac{4}{3^2\pi} + \frac{4}{5^2 \pi} + \cdots\right)\\ 
    &\iff& \boxed{\frac{\pi^2}{8}} &= 1 + \frac{1}{3^2} + \frac{1}{5^2} + \frac{1}{7^2}+\cdots
    \end{align*}
    
\end{proof}

\begin{problem} % 2
    Consider the boundary value problem
    \begin{align*}
        &f'' + \lambda f = 0 &  &0 \leq x \leq \pi \\
        &f(0) = 0 && f(\pi) -2f'(0) = 0
    \end{align*}
    
    \begin{enumerate}[label=\alph*.)]
        \item One of the hypotheses of a SL BVP with separated boundary conditions doesn't hold. Which one?
        \item Is $\lambda =0$ an e-value?
        \item Are there any e-values for $\lambda < 0$? If so find them.
        \item Are there any e-values for $\lambda > 0$? If so find them.
    \end{enumerate}
\end{problem}

\begin{proof}[\textbf{Solution}] % 2
\begin{enumerate}[label=\alph*.)]
    \item TBD
    
    
    \item If $\lambda = 0$, then $f = \alpha x + \beta$ and $f' = \alpha$.  \begin{align*}
    \intertext{ Since $f(0)=0$,}
        \alpha \cdot 0+\beta = 0 \implies &\beta = 0  &\text{so}  &  &f=\alpha x \text{ and } f' = \alpha
     \intertext{Since $f(\pi) - 2f'(0) = 0$,}
        \alpha \pi -2\alpha = \alpha(\pi-2) = 0 \implies &\alpha = 0 &\text{so} &  &f \equiv 0
    \end{align*}
    Since $f\equiv 0$ cannot be an eigenfunction, \underline{$\lambda = 0$ is not an eigenvalue for the SL-BVP}.
    
    \item  \underline{For $\lambda < 0$}, let $\lambda = -a^2$. Then $f = C_1 \cosh (a x) + C_2 \sinh (ax)$ and \\ \hspace*{2.06in} $f' =  C_1 a\sinh (a x) + C_2 a\cosh (ax)$.
    \begin{align*}
    \intertext{Since $f(0)=0$,}
        C_1 \cdot 1 + C_2 \cdot 0 = 0 \implies &C_1 = 0 &\text{so} &  &\begin{cases}
        f=C_2 \sinh(ax), \\ f'=C_2a\cosh(ax)\end{cases}
    \intertext{Since $f(\pi) - 2f'(0) = 0$,}
        \cancel{C_2} \sinh(a\pi) - 2\cancel{C_2} a\cdot 1 = 0 \implies &\sinh(a\pi) = 2a &&&
    \end{align*}
    Therefore, the eigenvalues $\lambda < 0$ will be the solutions to the transcendental equation $\sinh (a \pi) = 2a$, $a=\sqrt{-\lambda}$. To find these solutions, we consider the graphs of both sides of the equation, $y_1 = \sinh (a\pi)$ and $y_2 = 2a$. Since at  $a=0$ (outside the domain of consideration), $y_1(0) = y_2(0)=0$, the origin is contained in graphs of both. For $a>0$, we note that $y'_1 = \pi \cosh (a \pi) > \pi >  y'_2 = 2$. Since $y_1$ always increases faster than $y_2$ on the interval, there will be no intersections for $a>0$. Therefore, \underline{there are no eignenvalues  $\lambda < 0$}. This result is confirmed graphically by plotting $y_1$ and $y_2$:
    
    \item \underline{For $\lambda > 0$}, \begin{equation*}
        f=C_1 \cos \left(\sqrt{\lambda}x\right) + C_2 \sin\left(\sqrt{\lambda}x\right), \qquad
        f'=-C_1 \sqrt{\lambda}\sin \left(\sqrt{\lambda}x\right) + C_2 \sqrt{\lambda}\cos\left(\sqrt{\lambda}x\right)
    \end{equation*}
\end{enumerate}
\end{proof}


\begin{problem} % 3
    Consider the boundary value problem
    \begin{align*}
    &f'' + \lambda f = 0 &  &0 \leq x \leq \pi \\
    &f(0) + f'(0) = 0 && f(\pi) -f'(\pi) = 0
    \end{align*}
    
    \begin{enumerate}[label=\alph*.)]
        \item Determine if there are \underline{negative} e-values, and if so display them graphically and estimate.
        \item State what happens to the negative e-value(s) of this SL BVP if the right end point is $p$ where $p>\pi$.
    \end{enumerate}
\end{problem}

\end{document}

