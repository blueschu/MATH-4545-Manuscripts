\documentclass[letterpaper,11pt]{article} % Copyright (c) 2020  Brian Schubert
\usepackage{amsmath,amsthm,amsfonts,amssymb}
\usepackage{mathtools}
\usepackage{titling}
\usepackage{enumitem}
\usepackage{geometry}
\usepackage[makeroom]{cancel}

\usepackage{framed}

\geometry{left=0.75in,right=0.75in,top=1in,bottom=1in}

\theoremstyle{plain}
\newtheorem{problem}{Problem}

% Hide QED symbol
%\renewcommand{\qedsymbol}{}

\begin{document}
% Title
\begin{center}
    \huge
    Math 4545 Take Home \#2

    \vspace{0.5cm}
    \large
    Brian Schubert

    \vspace{0.5cm}
    Due: 04/14/2020
\end{center}

\begin{problem} % Problem 1
    \hfill
    \begin{enumerate}[label=\alph*.)]
        \item Read Constanda 5.1.5, do 5.1.5\#1,4:\\
        \fbox{\hspace{0.02\textwidth}\parbox{0.88\textwidth}{
            Use separation of variables to solve the IBVP
            \begin{gather*}
                \mu_t (x,t) = \mu_{xx}(x,t), \qquad -1 < x < 1, \quad t>0, \\
                \mu(-1, t) = \mu(1,t), \quad \mu_x(-1, t) = \mu(1,t), \quad t > 0, \\
                \mu(x, 0) = f(x), \quad -1 < x < 1,
            \end{gather*}
            with function $f$ as indicated.
            \begin{enumerate}[label=\#\arabic*.)]
                \item $f(x) = 2\sin (2\pi x) - \cos(5\pi x)$.
                \addtocounter{enumii}{2} % Skip to question 4
                \item $f(x) = \begin{cases} 2, & -1 < x \leq 0, \\ x -1, & \phantom{-}0 < x < 1. \end{cases}$
            \end{enumerate}
        }}
        \item Read Constanda 5.3.2, do 5.3.2\#1,4:\\
        \fbox{\hspace{0.02\textwidth}\parbox{0.88\textwidth}{
                Use separation of variables to solve the BVP
                \begin{gather*}
                \mu_{rr}(r, \theta) + r^{-1}\mu_r(r, \theta) + r^{-2} \mu_{\theta\theta}(r, \theta) = 0, \quad 0 < r < \alpha, \quad -\pi < \theta < \pi, \\
                \mu(r, \theta), \mu_r(r, \theta) \text{ bounded as } r\to 0^+, \quad \mu(\alpha, \theta) = f(\theta), \quad -\pi < \theta < \pi, \\
                \mu(r, -\pi) = \mu(r,\pi), \quad \mu_{\theta}(r, -\pi) - \mu_\theta(r, \pi), \quad 0 < r < \alpha.
                \end{gather*}
                with constant $\alpha$ and function $f$ as indicated.
                \begin{enumerate}[label=\#\arabic*.)]
                    \item $\alpha = 1/2, \quad f(\theta) = 3-4\cos(3\theta)$.
                    \addtocounter{enumii}{2} % Skip to question 4
                    \item $\alpha =1, \quad f(x) = \begin{cases} 1, & -\pi < \theta \leq 0, \\ \theta, & \phantom{-}0 < \theta < \pi. \end{cases}$
                \end{enumerate}
        }}
    \end{enumerate}
\end{problem}


\begin{proof}[\textbf{Solution 5.1.5\#1}] % Problem 1a (#1)
   Let $\mu(x,t) = X(x)T(t)$. Then
   \begin{gather*}
    X(x)T'(t) = X''(x)T(t) \implies \frac{T'(t)}{T(t)} = \frac{X''(x)}{X(x)} = -\lambda
    \implies \left\{\begin{array}{l}
    X''(x) + \lambda X(x) = 0 \\ T'(t) + \lambda T = 0
    \end{array}\right.
    \intertext{with boundary conditions}
    X(-1) = X(1), \qquad X'(-1) = X(1).
   \end{gather*}
   The eigenvalues for the given S-L problem are
   \begin{gather*}
   \lambda_0 = 0, \quad \lambda_n = \left(n \pi \right)^2, \quad n \in \mathbb{Z^+}
   \intertext{with corresponding eigenfunctions}
   X_0(x) = \frac{1}{2}, \quad X_{1n}(x) = \cos n \pi x, \quad X_{2n} = \sin n \pi x, \quad n \in \mathbb{Z^+}.
   \end{gather*}
   The function $\mu$ can be expressed as an infinite series of the eigenfunctions as
   \begin{equation*}
   \mu(x,t) = \frac{1}{2} a_0 + \sum_{n=1}^\infty \Big( a_n \cos \left(n \pi x\right) + b_n \sin \left( n\pi x \right) \Big) e^{-n^2 \pi^2 t}.
   \end{equation*}
   Since $\mu(x, 0) = f(x) = 2\sin (2\pi x) - \cos(5\pi x)$, we have
   \begin{equation*}
   \mu(x,t) = \frac{1}{2} a_0 + \sum_{n=1}^\infty \Big( a_n \cos \left(n \pi x\right) + b_n \sin \left( n\pi x \right) \Big) = 2\sin (2\pi x) - \cos(5\pi x).
   \end{equation*}
   The right hand side is itself a linear combination of the eigenfunctions, so the coefficients $a_n$ and $b_n$ can be found by match-up to be
   \begin{equation*}
   a_0 = 0, \qquad a_n = \begin{cases}
   2, & n=2, \\ 0, & \text{otherwise}
   \end{cases}, \qquad 
   b_n = \begin{cases}
   -1, & n=5, \\ 0, & \text{otherwise}
   \end{cases}, \qquad 
   n \in \mathbb{Z^+}
   \end{equation*}
   Therefore the complete solution is
   \begin{equation*}
   \boxed{\mu(x,t) = 2\sin (2\pi x)e^{-4\pi^2 t} - \cos(5\pi x)e^{-25 \pi^2 t}.}
   \end{equation*}
\end{proof} 


\begin{proof}[\textbf{Solution 5.1.5\#4}] % Problem 1a (#4)
    Using the same setup as 5.1.5\#1, we have \begin{equation*}
    \mu(x,t) = \frac{1}{2} a_0 + \sum_{n=1}^\infty \Big( a_n \cos \left(n \pi x\right) + b_n \sin \left( n\pi x \right) \Big) e^{-n^2 \pi^2 t}.
    \end{equation*}
     Since $\mu(x, 0) = f(x)$, we have
    \begin{equation*}
    \mu(x,t) = \frac{1}{2} a_0 + \sum_{n=1}^\infty \Big( a_n \cos \left(n \pi x\right) + b_n \sin \left( n\pi x \right) \Big) = \begin{cases} 2, & -1 < x \leq 0, \\ x -1, & \phantom{-}0 < x < 1. \end{cases}
    \end{equation*}
    Which is the full Fourier series of $f$. The coefficients $a_n$ and $b_n$ are given by
    \begin{align*}
        &a_0 = \int_{-1}^1 f(x) \, \mathrm dx& &= 2\int_{-1}^0 \!\mathrm{d}x + \int_0^1 (x-1) \, \mathrm{d}x = 2 + \left.\left(\frac{1}{2}x^2 - x\right)\right|_0^1 = \boxed{\frac{3}{2}}. \\
        &a_n= \int_{-1}^1 f(x) \cos n \pi x \, \mathrm{d}x& &= \cancelto{0}{2 \int_{-1}^0 \cos(n\pi x)\,\mathrm{d}x} + \int_0^1 (x-1) \cos (n \pi x) \,\mathrm{d}x \\
        &&&= \cancelto{0}{\left.\frac{1}{n\pi}(x-1)\sin(n\pi x)\right|_0^1}  - \frac{1}{n\pi}\int_0^1 \sin(n \pi x) \, \mathrm dx \\
        &&& = \left.\frac{1}{n^2 \pi^2} \cos (n\pi x) \right|_0^1 = \frac{1}{n^2 \pi^2}\left(\cos (n \pi) - 1\right) = \boxed{\frac{1}{n^2 \pi^2} \left( (-1)^{n} - 1\right).}\\
        &b_n = \int_{-1}^1 f(x) \sin n \pi x \, \mathrm{d}x& &= 2 \int_{-1}^0 \sin(n\pi x)\,\mathrm{d}x + \int_0^1 (x-1) \sin (n \pi x) \,\mathrm{d}x \\
        &&& = \left.-\frac{2}{n \pi} \cos(n\pi x)\right|_{-1}^0 + \left(\left.\frac{-1}{n\pi}(x-1)\cos(n\pi x)\right|_0^1 + \frac{1}{n\pi}\int_0^1 \cos(n\pi x)\,\mathrm{d}x\right) \\
        &&&=\frac{2}{n\pi} \left(\cos(n\pi) -1 \right) + \left(\frac{-1}{n\pi} + \cancelto{0}{\left.\frac{1}{n^2 \pi^2} \sin(n\pi x) \right|_0^1}\right) \\
        &&&=\boxed{\frac{2}{n\pi}\left((-1)^n - 1\right) - \frac{1}{n\pi}}
    \end{align*}
    Therefore, the complete solution is
    \begin{equation*}
    \boxed{\mu(x,t) = \frac{3}{4} + \sum_{n=1}^\infty \left( \frac{1}{n^2 \pi^2} \Big( (-1)^{n} - 1\Big)\cos(n\pi x) + \left(\frac{2}{n\pi}\Big((-1)^n - 1\Big) - \frac{1}{n\pi}\right) \sin(n\pi x) \right)e^{-n^2 \pi^2 t}}
    \end{equation*}
\end{proof}


\begin{proof}[\textbf{Solution 5.3.2\#1}] % Problem 1b (#1)
    Let $\mu(r, \theta) = R(r)\Theta(\theta)$, which yields the Sturm-Liouville problem
    \begin{gather*}
    \Theta''(\theta) + \lambda \Theta(\theta) = 0, \qquad -\pi < \theta < \pi, \\
    \Theta(-\pi) = \Theta(\pi), \qquad \Theta'(-\pi) = \Theta' (\pi).
    \end{gather*}
    The eigenvalues of this S-L problem are
    \begin{gather*}
    \lambda_0 = 0, \qquad \lambda_n = n^2, \qquad n \in \mathbb{Z}^+
    \intertext{with corresponding eigenfunctions}
    \Theta_0(\theta) = 1, \quad \Theta_{1n} = \cos(n\theta), \quad \Theta_{2n} = \sin (n\theta), \quad n \in \mathbb{Z}^+.
    \end{gather*}
    The general solution is of the form
    \begin{equation*}
    \mu(r, \theta) = \frac{1}{2} a_0 + \sum_{n=1}^\infty r^n \Big(a_n \cos(n\theta) + b_n \sin(n\theta) \Big).
    \end{equation*}
    Since $\mu(1/2, \theta) = f(\theta) = 3 - 4\cos(3\theta)$, we have
    \begin{equation*}
        \mu\left(\frac{1}{2}, \theta\right) = \frac{1}{2} a_0 + \sum_{n=1}^\infty \left(\frac{1}{2}\right)^n \Big(a_n \cos(n\theta) + b_n \sin(n\theta) \Big) = 3 - 4\cos(3\theta).
    \end{equation*}
    The coefficients $a_0$, $a_n$, and $b_n$ can be determined by match-up to be
    \begin{equation*}
    a_0 = 6, \qquad a_n = \begin{cases} -32, & n = 3, \\ 0, &\text{otherwise}.\end{cases}, \qquad b_n = 0.
    \end{equation*}
    Therefore the complete solution is
    \begin{equation*}
    \boxed{\mu(r, \theta) = 3 - 32r^3 \cos(3\theta)}
    \end{equation*}
\end{proof}


\begin{proof}[\textbf{Solution 5.3.2\#4}] % Problem 1b (#4)
    Using the same setup as 5.2.3\#1 above, we have
    \begin{equation*}
    \mu(r, \theta) = \frac{1}{2} a_0 + \sum_{n=1}^\infty r^n \Big(a_n \cos(n\theta) + b_n \sin(n\theta) \Big).
    \end{equation*}
     Since $\mu(1, \theta)$, we have
    \begin{equation*}
    \mu(1, \theta) = \frac{1}{2} a_0 + \sum_{n=1}^\infty 1^n \Big(a_n \cos(n\theta) + b_n \sin(n\theta) \Big) =\begin{cases} 1, & -\pi < \theta \leq 0, \\ \theta, & \phantom{-}0 < \theta < \pi. \end{cases}.
    \end{equation*}
     which is the full Fourier series for $f$. The coefficients $a_0$, $a_n$, and $b_n$ can therefore be computed by
     \begin{align*}
        &a_0 = \frac{1}{\pi} \int_{-\pi}^\pi f(\theta)\, \mathrm{d}\theta& &=\frac{1}{\pi}\left( \int_{-\pi}^0\!\mathrm{d}\theta + \int_0^\pi \theta \,\mathrm{d}\theta \right)= \frac{1}{\pi}\left(\pi + \left.\frac{1}{2} \theta^2 \right|_0^\pi \right)= \frac{1}{\pi}\left(\pi + \frac{\pi^2}{2}\right) = \boxed{1 + \frac{\pi}{2}}.\\
        &a_n = \frac{1}{\pi 1^n} \int_{-\pi}^{\pi} f(\theta) \cos(n\theta)\,\mathrm{d}\theta& &= \frac{1}{\pi} \left( \cancelto{0}{\int_{-\pi}^0 \cos(n\theta) \,\mathrm{d}\theta} + \int_0^\pi \theta\cos(n\theta) \, \mathrm{d}\theta\right) \\ &&&= \frac{1}{\pi}\left(\cancelto{0}{\left.\frac{\theta}{n} \sin (n\theta) \right|_0^\pi} - \frac{1}{n}\int_0^\pi \sin (n\theta) \,\mathrm{d}\theta\right) = \frac{1}{\pi} \left( \frac{1}{n^2} \cos(n\theta)\right)_0^\pi \\
        &&&= \frac{1}{\pi}\left(\frac{1}{n^2} \Big(\cos(n\pi) - 1\Big)\right) = \boxed{\frac{1}{n^2 \pi}\left((-1)^n -1 \right)}.\\
        &b_n = \frac{1}{\pi 1^n} \int_{-\pi}^{\pi} f(\theta) \sin(n\theta)\,\mathrm{d}\theta& &= \frac{1}{\pi} \left(\int_{-\pi}^0 \sin(n\theta) \,\mathrm{d}\theta + \int_0^\pi \theta\sin(n\theta) \, \mathrm{d}\theta\right) \\
        &&&=\frac{1}{\pi} \left( \left.\frac{-1}{n} \cos(n \theta)\right|_{-\pi}^0 + \left.\frac{-\theta}{n}\cos(n\theta)\right|_0^\pi - \frac{1}{n} \int_0^\pi \cos(n\theta)\,\mathrm{d}\theta\right) \\
        &&&= \frac{1}{\pi}\left(\frac{1}{n}\Big(\cos(-n\pi) -1\Big) + \frac{-1}{n}\Big(\pi\cos(n\pi) - 0\Big) - \cancelto{0}{\left.\frac{1}{n^2} \sin(n\theta) \right|_0^\pi}\right) \\
        &&&= \frac{1}{\pi} \left(\frac{1}{n} \Big((-1)^n - 1\Big) + \frac{\pi}{n}(-1)^{n+1}\right) = \boxed{\frac{1}{n\pi} \Big((-1)^n - 1\Big) + \frac{(-1)^{n+1}}{n}}.
     \end{align*}
     Therefore, the complete solution is
      \begin{equation*}
     \boxed{\mu(r, \theta) = \frac{1}{2} + \frac{\pi}{4} + \sum_{n=1}^\infty r^n \left(\frac{1}{n^2 \pi}\Big((-1)^n -1 \Big) \cos(n\theta) + \left(\frac{1}{n\pi} \Big((-1)^n - 1\Big) + \frac{(-1)^{n+1}}{n}\right) \sin(n\theta) \right).}
     \end{equation*}
\end{proof}


    
\begin{problem} % Problem 2
    Given the following heat problem in a rod of length 1,
    \begin{gather*}
        \mu_t = \mu_{xx}, \qquad 0 < x< 1, \quad t > 0, \\
        \mu(0, t) = 0, \qquad \mu(1, t) + \mu_x(1,t) = 0,\\
        \mu(x, 0) = f(x) = 75.
    \end{gather*}
    \begin{enumerate}[label=\alph*.)]
        \item Find $\mu(x,t)$.
        \item Find $\lim\limits_{t \to \infty} \mu(x, t)$.
        \item Repeat a) and b) but change the right boundary condition to $\mu(1,t) -\mu_x(1,t) = 0$.
        \item You should be finding a significant change in your answers with the new right boundary condition. Please explain. Your explanation should address directly the effect of the sign change on your results.
    \end{enumerate}
\end{problem}

\begin{proof}[\textbf{Solution}] % Problem 2
    \hfill
    \begin{enumerate}[label=\alph*.)]
        \item Let $\mu(x, t) = X(x)T(t)$. Since the initial condition $f$ is non-homogeneous, neither $X$ nor $T$ can be the zero function. Substituting into the given heat equation produces 
        \begin{align*}
            X(x)T'(t) = X''(x)T(t) &\implies \frac{X''(x)}{X(x)} = \frac{T'(t)}{T(t)} = -\lambda &&\implies \left\{\begin{array}{l}
                X''(x) + \lambda X(x) = 0,\\
                T'(t) + \lambda T(t) = 0,
            \end{array}\right.
            \intertext{with boundary conditions}
            \mu(0,t) &\implies X(0)T(t) = 0 &&\implies X(0) = 0, \\
            \mu(1, t) + \mu_x(1, t) &\implies X(1)T(t) + X'(1)T(t) = 0 &&\implies X(1) + X'(1).
        \end{align*}
        
        This is a regular Sturm-Liouville boundary value problem in $X$. 
       
        \underline{If $\lambda = 0$,} then $X = \alpha x + \beta$ and $X' = \alpha$.  \begin{align*}
        \shortintertext{Since $X(0)=0$,}
        \alpha \cdot 0+\beta = 0 \implies &\beta = 0  &\text{so}  &  &X=\alpha x \text{ and } X' = \alpha.
        \shortintertext{Since $X(1) + X'(1) = 0$,}
        1\alpha + \alpha = 0 \implies & \alpha = 0 &\text{so} && X \equiv 0.
        \end{align*}
        Since $X\equiv 0$ cannot be an eigenfunction, \underline{$\lambda = 0$ is not an eigenvalue for the SL-BVP}.
        
        \underline{If $\lambda > 0$,} \begin{equation*}
        X=C_1 \cos \left(\sqrt{\lambda}x\right) + C_2 \sin\left(\sqrt{\lambda}x\right), \qquad
        X'=-C_1 \sqrt{\lambda}\sin \left(\sqrt{\lambda}x\right) + C_2 \sqrt{\lambda}\cos\left(\sqrt{\lambda}x\right)
        \end{equation*}
        \begin{gather*}
        \shortintertext{Since $X(0)=0$,}
        C_1 + 0 = 0 \implies  C_1 = 0  \qquad\text{so} \qquad \begin{cases}
                X=C_2 \sin(\sqrt{\lambda}x), \\ X'=C_2\sqrt{\lambda}\cos(\sqrt{\lambda}x)\end{cases}
        \shortintertext{Since $X(1) + X'(1) = 0$ and $C_2 \neq 0$,}
        C_2\sin(\sqrt{\lambda}) + C_2\sqrt{\lambda}\cos(\sqrt{\lambda}) = 0 \quad \implies\quad \tan(\sqrt{\lambda}) = -\sqrt{\lambda}
        \end{gather*}
        Therefore, the eigenvalues $\lambda_n > 0$ are the solutions to the transcendental equation $\tan(\sqrt{\lambda_n}) = -\sqrt{\lambda_n}$ with corresponding eigenfunctions $X \sim \sin(\sqrt{\lambda_n}x)$.
        
        \underline{If $\lambda < 0$},  let $\lambda = -a^2$ with $a>0$. Then
        \begin{equation*}
        X = C_1 \cosh (a x) + C_2 \sinh (ax) \quad\text{and}\quad X' =  C_1 a\sinh (a x) + C_2 a\cosh (ax).
        \end{equation*}
        \begin{align*}
        \shortintertext{Since $X(0)=0$,}
        &&X(0) = C_1 \cosh (0) + C_2 \sinh (0) &= 0\\
        &\iff& C_1 \cdot 1 + C_2 \cdot 0 &= 0 \\
        &\iff& C_1 &= 0 &\text{so} &  &\begin{cases}
        X=C_2 \sinh(ax), \\ X'=C_2a\cosh(ax)\end{cases}
        \shortintertext{Since $X(1) + X'(1) = 0$ and $C_2 \neq 0$,}
        &&\cancel{C_2} \sinh(a) + \cancel{C_2} a \cosh(a) &= 0\\
        &\iff& \tanh(a) &= -a\\
        \end{align*}
        Since the transcendental equation $\tanh(a) = -a$ has no solutions for $a>0$, there are no eigenvalues $\lambda <0$. 

        The formal solution to the given heat problem can therefore be expressed as
        \begin{equation*}
        \mu(x, t) = \sum_{n=1}^\infty c_n \sin(\sqrt{\lambda_n}x) e^{-\lambda_n t}.
        \end{equation*}
        Since $\mu(x, 0) = f(x) = 75$,
        \begin{align*}
            \mu(x, 0) = \sum_{n=1}^\infty c_n \sin(\sqrt{\lambda_n}x) &= f(x) = 75
            \shortintertext{so}
            c_n = \frac{\displaystyle \int_0^1 f(x) X_n(x)\,\mathrm{d}x}{\displaystyle\int_0^1 X^2_n(x)\,\mathrm{d}x} &= \frac{\displaystyle 75\int_0^1  \sin(\sqrt{\lambda_n}x)\,\mathrm{d}x}{\displaystyle\int_0^1 \sin^2(\sqrt{\lambda_n}x)\,\mathrm{d}x} = \frac{\displaystyle \frac{75}{\sqrt{\lambda_n}}\left.\cos(\sqrt{\lambda_n}x)\right|_0^1}{\displaystyle \int_0^1 \left(\frac{1}{2} - \frac{1}{2} \cos(2\sqrt{\lambda_n}x)\right)\,\mathrm{d}x} \\
            &= \frac{\displaystyle \frac{75}{\sqrt{\lambda_n}}\left(\cos(\sqrt{\lambda_n} )- 1\right)}{\displaystyle \frac{1}{2} - \frac{1}{4\sqrt{\lambda_n}}\left.\sin(2\sqrt{\lambda_n}x)\right|_0^1 } = \frac{\displaystyle \frac{75}{\sqrt{\lambda_n}}\left(\cos(\sqrt{\lambda_n} )- 1\right)}{\displaystyle \frac{1}{2} - \frac{1}{4\sqrt{\lambda_n}}\sin(2\sqrt{\lambda_n})} \tag{$\ast$}
        \end{align*}
        The complete solution is therefore
        \begin{equation*}
         \boxed{\mu(x, t) = \sum_{n=1}^\infty \left(\frac{\displaystyle 75\left(\cos(\sqrt{\lambda_n} )- 1\right)}{\displaystyle \frac{\sqrt{\lambda_n}}{2} - \frac{1}{4}\sin(2\sqrt{\lambda_n})} \right)\sin(\sqrt{\lambda_n}x) e^{-\lambda_n t}}.
        \end{equation*}
        where the eigenvalues $\lambda_n$ satisfy the transcendental equation $\tan(\sqrt{\lambda_n}) = -\sqrt{\lambda_n}$.
        
        \item \begin{equation*}
        \lim\limits_{t \to \infty} \mu(x, t) =  \lim\limits_{t \to \infty}\sum_{n=1}^\infty c_n \sin(\sqrt{\lambda_n}x) e^{-\lambda_n t} \geq \sum_{n=1}^\infty \lim\limits_{t \to \infty} c_n \sin(\sqrt{\lambda_n}x) e^{-\lambda_n t} = \sum_{n=1}^\infty 0 = \boxed{0}.
        \end{equation*}
        
        \item Using the same setup as part a), we produce the regular SL-BVP
        \begin{gather*}
            X''(x) + \lambda X(x) = 0, \qquad 0 < x < 1, \\
            X(0) = 0, \qquad X(1) - X'(1) = 0.
        \end{gather*}
         \underline{If $\lambda = 0$,} then $X = \alpha x + \beta$ and $X' = \alpha$.  \begin{align*}
        \shortintertext{Since $X(0)=0$,}
        \alpha \cdot 0+\beta = 0 \implies &\beta = 0  &\text{so}  &  &X=\alpha x \text{ and } X' = \alpha.
        \shortintertext{Since $X(1) - X'(1) = 0$,}
        1\alpha - \alpha = 0 \implies & \alpha(1-1) = 0 &\text{so} &&  X=\alpha x \text{ and } X' = \alpha.
        \end{align*}
        Therefore $\lambda = 0$ is an eigenvalue with corresponding eigenfunctions $X_0\sim x$.
        
        \underline{If $\lambda > 0$,} \begin{equation*}
        X=C_1 \cos \left(\sqrt{\lambda}x\right) + C_2 \sin\left(\sqrt{\lambda}x\right), \qquad
        X'=-C_1 \sqrt{\lambda}\sin \left(\sqrt{\lambda}x\right) + C_2 \sqrt{\lambda}\cos\left(\sqrt{\lambda}x\right)
        \end{equation*}
        \begin{gather*}
        \shortintertext{Since $X(0)=0$,}
        C_1 + 0 = 0 \implies  C_1 = 0  \qquad\text{so} \qquad \begin{cases}
        X=C_2 \sin(\sqrt{\lambda}x), \\ X'=C_2\sqrt{\lambda}\cos(\sqrt{\lambda}x)\end{cases}
        \shortintertext{Since $X(1) - X'(1) = 0$ and $C_2 \neq 0$,}
        C_2\sin(\sqrt{\lambda}) - C_2\sqrt{\lambda}\cos(\sqrt{\lambda}) = 0 \quad \implies\quad \tan(\sqrt{\lambda}) = \sqrt{\lambda}
        \end{gather*}
        Therefore, the eigenvalues $\lambda_n > 0$ are the solutions to the transcendental equation $\tan(\sqrt{\lambda_n}) = \sqrt{\lambda_n}$ with corresponding eigenfunctions $X_n \sim \sin(\sqrt{\lambda_n}x)$.
        
        \underline{If $\lambda < 0$},  let $\lambda = -a^2$ with $a>0$. Then
        \begin{equation*}
        X = C_1 \cosh (a x) + C_2 \sinh (ax) \quad\text{and}\quad X' =  C_1 a\sinh (a x) + C_2 a\cosh (ax).
        \end{equation*}
        \begin{align*}
        \shortintertext{Since $X(0)=0$,}
        &&X(0) = C_1 \cosh (0) + C_2 \sinh (0) &= 0\\
        &\iff& C_1 \cdot 1 + C_2 \cdot 0 &= 0 \\
        &\iff& C_1 &= 0 &\text{so} &  &\begin{cases}
        X=C_2 \sinh(ax), \\ X'=C_2a\cosh(ax)\end{cases}
        \shortintertext{Since $X(1) - X'(1) = 0$ and $C_2 \neq 0$,}
        &&\cancel{C_2} \sinh(a) - \cancel{C_2} a \cosh(a) &= 0\\
        &\iff& \tanh(a) &= a
        \end{align*}
        Therefore the eigenvalues $\lambda < 0$ will be the solution to the transcendental equation $ \tanh(a) = a$. To find these solutions, we consider the graphs of $y_1 = \tanh a$ and $y_2 = a$. We note that at $a=0$, $y_1 = y_2$. For $a>0$, we note that $y'_1 = 1 - \tanh^2 a < 1 = y_2'$. Therefore, since $y_2$ always grows faster than $y_1$, there are no further intersections, and so the equation $\tanh a = a$ has no solution for $a>0$. Therefore, there are no eigenvalues $\lambda < 0$.
        
         The formal solution to the given heat problem can therefore be expressed as
        \begin{equation*}
        \mu(x, t) = c_0x + \sum_{n=1}^\infty c_n \sin(\sqrt{\lambda_n}x) e^{-\lambda_n t}.
        \end{equation*}
        
        Since $\mu(x, 0) = f(x) = 75$,
        \begin{align*}
        \mu(x, 0) = c_0 x + \sum_{n=1}^\infty c_n \sin(\sqrt{\lambda_n}x) &= f(x) = 75
        \end{align*}
        we can compute the coefficients $c_0$ and $c_n$ by expressing $f$ as an infinite sum of the eigenfunctions:
        \begin{gather*}
        c_0 = \frac{\displaystyle \int_0^1 f(x) X_0(x)\,\mathrm{d}x}{\displaystyle\int_0^1 X^2_0(x)\,\mathrm{d}x} = \frac{75\displaystyle \int_0^1 x\,\mathrm{d}x}{\displaystyle\int_0^1 x^2\,\mathrm{d}x} = \frac{75/2}{1/3} =\frac{75}{6}
        \end{gather*}
        and with $c_n$s computed as seen in ($\ast$) above.
        
        The complete solution is therefore
        \begin{equation*}
        \boxed{\mu(x, t) = \frac{75}{6}x + \sum_{n=1}^\infty \left(\frac{\displaystyle 75\left(\cos(\sqrt{\lambda_n} )- 1\right)}{\displaystyle \frac{\sqrt{\lambda_n}}{2} - \frac{1}{4}\sin(2\sqrt{\lambda_n})} \right)\sin(\sqrt{\lambda_n}x) e^{-\lambda_n t}}.
        \end{equation*}
        where the eigenvalues $\lambda_n$ satisfy the transcendental equation $\tan(\sqrt{\lambda_n}) = \sqrt{\lambda_n}$. Computing the limit as $t\to \infty$ produces
         \begin{align*}
        \lim\limits_{t \to \infty} \mu(x, t) &=  \lim\limits_{t \to \infty}\left(\frac{75}{6}x + \sum_{n=1}^\infty c_n \sin(\sqrt{\lambda_n}x) e^{-\lambda_n t}\right) = \lim\limits_{t \to \infty}\frac{75}{6}x + \lim\limits_{t \to \infty}\sum_{n=1}^\infty c_n \sin(\sqrt{\lambda_n}x) e^{-\lambda_n t} \\
         &= \frac{75}{6}x + \lim\limits_{t \to \infty}\sum_{n=1}^\infty c_n \sin(\sqrt{\lambda_n}x) e^{-\lambda_n t}
         \geq \frac{75}{6}x + \sum_{n=1}^\infty \lim\limits_{t \to \infty} c_n \sin(\sqrt{\lambda_n}x) e^{-\lambda_n t}\\
         &= \frac{75}{6}x + \sum_{n=1}^\infty 0 = \boxed{\frac{75}{6}x}.
        \end{align*}
        
        \item test
    \end{enumerate}
\end{proof}


\end{document}

