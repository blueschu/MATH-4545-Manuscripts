\documentclass[letterpaper,11pt]{article} % Copyright (c) 2020  Brian Schubert
\usepackage{amsmath,amsthm,amsfonts,amssymb}
\usepackage{mathtools}
\usepackage{titling}
\usepackage{enumitem}
\usepackage{geometry}
\usepackage[makeroom]{cancel}

\geometry{left=0.75in,right=0.75in,top=1in,bottom=1in}

\theoremstyle{plain}
\newtheorem{problem}{Problem}

% Hide QED symbol
%\renewcommand{\qedsymbol}{}

\begin{document}
% Title
\begin{center}
    \huge
    Math 4545 Take Home \#2

    \vspace{0.5cm}
    \large
    Brian Schubert

    \vspace{0.5cm}
    Due: 04/14/2020
\end{center}

\begin{problem} % Problem 1
    \hfill
    \begin{enumerate}[label=\alph*.)]
        \item Read Constanda 5.1.5, do 5.1.5\#1,4:\\
        \fbox{\hspace{0.02\textwidth}\parbox{0.88\textwidth}{
            Use separation of variables to solve the IBVP
            \begin{gather*}
                \mu_t (x,t) = \mu_{xx}(x,t), \qquad -1 < x < 1, \quad t>0, \\
                \mu(-1, t) = \mu(1,t), \quad \mu_x(-1, t) = \mu(1,t), \quad t > 0, \\
                \mu(x, 0) = f(x), \quad -1 < x < 1,
            \end{gather*}
            with function $f$ as indicated.
            \begin{enumerate}[label=\#\arabic*.)]
                \item $f(x) = 2\sin (2\pi x) - \cos(5\pi x)$.
                \addtocounter{enumii}{2} % Skip to question 4
                \item $f(x) = \begin{cases} 2, & -1 < x \leq 0, \\ x -1, & \phantom{-}0 < x < 1. \end{cases}$
            \end{enumerate}
        }}
        \item Read Constanda 5.3.2, do 5.3.2\#1,4:\\
        \fbox{\hspace{0.02\textwidth}\parbox{0.88\textwidth}{
                Use separation of variables to solve the BVP
                \begin{gather*}
                \mu_{rr}(r, \theta) + r^{-1}\mu_r(r, \theta) + r^{-2} \mu_{\theta\theta}(r, \theta) = 0, \quad 0 < r < \alpha, \quad -\pi < \theta < \pi, \\
                \mu(r, \theta), \mu_r(r, \theta) \text{ bounded as } r\to 0^+, \quad \mu(\alpha, \theta) = f(\theta), \quad -\pi < \theta < \pi, \\
                \mu(r, -\pi) = \mu(r,\pi), \quad \mu_{\theta}(r, -\pi) - \mu_\theta(r, \pi), \quad 0 < r < \alpha.
                \end{gather*}
                with constant $\alpha$ and function $f$ as indicated.
                \begin{enumerate}[label=\#\arabic*.)]
                    \item $\alpha = 1/2, \quad f(\theta) = 3-4\cos(3\theta)$.
                    \addtocounter{enumii}{2} % Skip to question 4
                    \item $\alpha =1, \quad f(x) = \begin{cases} 1, & -\pi < \theta \leq 0, \\ \theta, & \phantom{-}0 < \theta < \pi. \end{cases}$
                \end{enumerate}
        }}
    \end{enumerate}
\end{problem}


\begin{proof}[\textbf{Solution 5.1.5\#1}] % Problem 1a (#1)
   Let $\mu(x,t) = X(x)T(t)$. Then
   \begin{gather*}
    X(x)T'(t) = X''(x)T(t) \implies \frac{T'(t)}{T(t)} = \frac{X''(x)}{X(x)} = -\lambda
    \implies \left\{\begin{array}{l}
    X''(x) + \lambda X(x) = 0 \\ T'(t) + \lambda T = 0
    \end{array}\right.
    \intertext{with boundary conditions}
    X(-1) = X(1), \qquad X'(-1) = X(1).
   \end{gather*}
   The eigenvalues for the given S-L problem are
   \begin{gather*}
   \lambda_0 = 0, \quad \lambda_n = \left(n \pi \right)^2, \quad n \in \mathbb{Z^+}
   \intertext{with corresponding eigenfunctions}
   X_0(x) = \frac{1}{2}, \quad X_{1n}(x) = \cos n \pi x, \quad X_{2n} = \sin n \pi x, \quad n \in \mathbb{Z^+}.
   \end{gather*}
   The function $\mu$ can be expressed as an infinite series of the eigenfunctions as
   \begin{equation*}
   \mu(x,t) = \frac{1}{2} a_0 + \sum_{n=1}^\infty \Big( a_n \cos \left(n \pi x\right) + b_n \sin \left( n\pi x \right) \Big) e^{-n^2 \pi^2 t}.
   \end{equation*}
   Since $\mu(x, 0) = f(x) = 2\sin (2\pi x) - \cos(5\pi x)$, we have
   \begin{equation*}
   \mu(x,t) = \frac{1}{2} a_0 + \sum_{n=1}^\infty \Big( a_n \cos \left(n \pi x\right) + b_n \sin \left( n\pi x \right) \Big) = 2\sin (2\pi x) - \cos(5\pi x).
   \end{equation*}
   The right hand side is itself a linear combination of the eigenfunctions, so the coefficients $a_n$ and $b_n$ can be found by match-up to be
   \begin{equation*}
   a_0 = 0, \qquad a_n = \begin{cases}
   2, & n=2, \\ 0, & \text{otherwise}
   \end{cases}, \qquad 
   b_n = \begin{cases}
   -1, & n=5, \\ 0, & \text{otherwise}
   \end{cases}, \qquad 
   n \in \mathbb{Z^+}
   \end{equation*}
   Therefore the complete solution is
   \begin{equation*}
   \boxed{\mu(x,t) = 2\sin (2\pi x)e^{-4\pi^2 t} - \cos(5\pi x)e^{-25 \pi^2 t}.}
   \end{equation*}
\end{proof} 


\begin{proof}[\textbf{Solution 5.1.5\#4}] % Problem 1a (#4)
    Using the same setup as 5.1.5\#1, we have \begin{equation*}
    \mu(x,t) = \frac{1}{2} a_0 + \sum_{n=1}^\infty \Big( a_n \cos \left(n \pi x\right) + b_n \sin \left( n\pi x \right) \Big) e^{-n^2 \pi^2 t}.
    \end{equation*}
     Since $\mu(x, 0) = f(x)$, we have
    \begin{equation*}
    \mu(x,t) = \frac{1}{2} a_0 + \sum_{n=1}^\infty \Big( a_n \cos \left(n \pi x\right) + b_n \sin \left( n\pi x \right) \Big) = \begin{cases} 2, & -1 < x \leq 0, \\ x -1, & \phantom{-}0 < x < 1. \end{cases}
    \end{equation*}
    Which is the full Fourier series of $f$. The coefficients $a_n$ and $b_n$ are given by
    \begin{align*}
        &a_0 = \int_{-1}^1 f(x) \, \mathrm dx& &= 2\int_{-1}^0 \!\mathrm{d}x + \int_0^1 (x-1) \, \mathrm{d}x = 2 + \left.\left(\frac{1}{2}x^2 - x\right)\right|_0^1 = \boxed{\frac{3}{2}}. \\
        &a_n= \int_{-1}^1 f(x) \cos n \pi x \, \mathrm{d}x& &= \cancelto{0}{2 \int_{-1}^0 \cos(n\pi x)\,\mathrm{d}x} + \int_0^1 (x-1) \cos (n \pi x) \,\mathrm{d}x \\
        &&&= \cancelto{0}{\left.\frac{1}{n\pi}(x-1)\sin(n\pi x)\right|_0^1}  - \frac{1}{n\pi}\int_0^1 \sin(n \pi x) \, \mathrm dx \\
        &&& = \left.\frac{1}{n^2 \pi^2} \cos (n\pi x) \right|_0^1 = \frac{1}{n^2 \pi^2}\left(\cos (n \pi) - 1\right) = \boxed{\frac{1}{n^2 \pi^2} \left( (-1)^{n} - 1\right).}\\
        &b_n = \int_{-1}^1 f(x) \sin n \pi x \, \mathrm{d}x& &= 2 \int_{-1}^0 \sin(n\pi x)\,\mathrm{d}x + \int_0^1 (x-1) \sin (n \pi x) \,\mathrm{d}x \\
        &&& = \left.-\frac{2}{n \pi} \cos(n\pi x)\right|_{-1}^0 + \left(\left.\frac{-1}{n\pi}(x-1)\cos(n\pi x)\right|_0^1 + \frac{1}{n\pi}\int_0^1 \cos(n\pi x)\,\mathrm{d}x\right) \\
        &&&=\frac{2}{n\pi} \left(\cos(n\pi) -1 \right) + \left(\frac{-1}{n\pi} + \cancelto{0}{\left.\frac{1}{n^2 \pi^2} \sin(n\pi x) \right|_0^1}\right) \\
        &&&=\boxed{\frac{2}{n\pi}\left((-1)^n - 1\right) - \frac{1}{n\pi}}
    \end{align*}
    Therefore, the complete solution is
    \begin{equation*}
    \boxed{\mu(x,t) = \frac{3}{4} + \sum_{n=1}^\infty \left( \frac{1}{n^2 \pi^2} \Big( (-1)^{n} - 1\Big)\cos(n\pi x) + \left(\frac{2}{n\pi}\Big((-1)^n - 1\Big) - \frac{1}{n\pi}\right) \sin(n\pi x) \right)e^{-n^2 \pi^2 t}}
    \end{equation*}
\end{proof}


\begin{proof}[\textbf{Solution 5.3.2\#1}] % Problem 1b (#1)
    Let $\mu(r, \theta) = R(r)\Theta(\theta)$, which yields the Sturm-Liouville problem
    \begin{gather*}
    \Theta''(\theta) + \lambda \Theta(\theta) = 0, \qquad -\pi < \theta < \pi, \\
    \Theta(-\pi) = \Theta(\pi), \qquad \Theta'(-\pi) = \Theta' (\pi).
    \end{gather*}
    The eigenvalues of this S-L problem are
    \begin{gather*}
    \lambda_0 = 0, \qquad \lambda_n = n^2, \qquad n \in \mathbb{Z}^+
    \intertext{with corresponding eigenfunctions}
    \Theta_0(\theta) = 1, \quad \Theta_{1n} = \cos(n\theta), \quad \Theta_{2n} = \sin (n\theta), \quad n \in \mathbb{Z}^+.
    \end{gather*}
    The general solution is of the form
    \begin{equation*}
    \mu(r, \theta) = \frac{1}{2} a_0 + \sum_{n=1}^\infty r^n \Big(a_n \cos(n\theta) + b_n \sin(n\theta) \Big).
    \end{equation*}
    Since $\mu(1/2, \theta) = f(\theta) = 3 - 4\cos(3\theta)$, we have
    \begin{equation*}
        \mu\left(\frac{1}{2}, \theta\right) = \frac{1}{2} a_0 + \sum_{n=1}^\infty \left(\frac{1}{2}\right)^n \Big(a_n \cos(n\theta) + b_n \sin(n\theta) \Big) = 3 - 4\cos(3\theta).
    \end{equation*}
    The coefficients $a_0$, $a_n$, and $b_n$ can be determined by match-up to be
    \begin{equation*}
    a_0 = 6, \qquad a_n = \begin{cases} -32, & n = 3, \\ 0, &\text{otherwise}.\end{cases}, \qquad b_n = 0.
    \end{equation*}
    Therefore the complete solution is
    \begin{equation*}
    \boxed{\mu(r, \theta) = 3 - 32r^3 \cos(3\theta)}
    \end{equation*}
\end{proof}


\begin{proof}[\textbf{Solution 5.3.2\#4}] % Problem 1b (#4)
    Using the same setup as 5.2.3\#1 above, we have
    \begin{equation*}
    \mu(r, \theta) = \frac{1}{2} a_0 + \sum_{n=1}^\infty r^n \Big(a_n \cos(n\theta) + b_n \sin(n\theta) \Big).
    \end{equation*}
     Since $\mu(1, \theta)$, we have
    \begin{equation*}
    \mu(1, \theta) = \frac{1}{2} a_0 + \sum_{n=1}^\infty 1^n \Big(a_n \cos(n\theta) + b_n \sin(n\theta) \Big) =\begin{cases} 1, & -\pi < \theta \leq 0, \\ \theta, & \phantom{-}0 < \theta < \pi. \end{cases}.
    \end{equation*}
     which is the full Fourier series for $f$. The coefficients $a_0$, $a_n$, and $b_n$ can therefore be computed by
     \begin{align*}
        &a_0 = \frac{1}{\pi} \int_{-\pi}^\pi f(\theta)\, \mathrm{d}\theta& &=\frac{1}{\pi}\left( \int_{-\pi}^0\!\mathrm{d}\theta + \int_0^\pi \theta \,\mathrm{d}\theta \right)= \frac{1}{\pi}\left(\pi + \left.\frac{1}{2} \theta^2 \right|_0^\pi \right)= \frac{1}{\pi}\left(\pi + \frac{\pi^2}{2}\right) = \boxed{1 + \frac{\pi}{2}}.\\
        &a_n = \frac{1}{\pi 1^n} \int_{-\pi}^{\pi} f(\theta) \cos(n\theta)\,\mathrm{d}\theta& &= \frac{1}{\pi} \left( \cancelto{0}{\int_{-\pi}^0 \cos(n\theta) \,\mathrm{d}\theta} + \int_0^\pi \theta\cos(n\theta) \, \mathrm{d}\theta\right) \\ &&&= \frac{1}{\pi}\left(\cancelto{0}{\left.\frac{\theta}{n} \sin (n\theta) \right|_0^\pi} - \frac{1}{n}\int_0^\pi \sin (n\theta) \,\mathrm{d}\theta\right) = \frac{1}{\pi} \left( \frac{1}{n^2} \cos(n\theta)\right)_0^\pi \\
        &&&= \frac{1}{\pi}\left(\frac{1}{n^2} \Big(\cos(n\pi) - 1\Big)\right) = \boxed{\frac{1}{n^2 \pi}\left((-1)^n -1 \right)}.\\
        &b_n = \frac{1}{\pi 1^n} \int_{-\pi}^{\pi} f(\theta) \sin(n\theta)\,\mathrm{d}\theta& &= \frac{1}{\pi} \left(\int_{-\pi}^0 \sin(n\theta) \,\mathrm{d}\theta + \int_0^\pi \theta\sin(n\theta) \, \mathrm{d}\theta\right) \\
        &&&=\frac{1}{\pi} \left( \left.\frac{-1}{n} \cos(n \theta)\right|_{-\pi}^0 + \left.\frac{-\theta}{n}\cos(n\theta)\right|_0^\pi - \frac{1}{n} \int_0^\pi \cos(n\theta)\,\mathrm{d}\theta\right) \\
        &&&= \frac{1}{\pi}\left(\frac{1}{n}\Big(\cos(-n\pi) -1\Big) + \frac{-1}{n}\Big(\pi\cos(n\pi) - 0\Big) - \cancelto{0}{\left.\frac{1}{n^2} \sin(n\theta) \right|_0^\pi}\right) \\
        &&&= \frac{1}{\pi} \left(\frac{1}{n} \Big((-1)^n - 1\Big) + \frac{\pi}{n}(-1)^{n+1}\right) = \boxed{\frac{1}{n\pi} \Big((-1)^n - 1\Big) + \frac{(-1)^{n+1}}{n}}.
     \end{align*}
     Therefore, the complete solution is
      \begin{equation*}
     \boxed{\mu(r, \theta) = \frac{1}{2} + \frac{\pi}{4} + \sum_{n=1}^\infty r^n \left(\frac{1}{n^2 \pi}\Big((-1)^n -1 \Big) \cos(n\theta) + \left(\frac{1}{n\pi} \Big((-1)^n - 1\Big) + \frac{(-1)^{n+1}}{n}\right) \sin(n\theta) \right).}
     \end{equation*}
\end{proof}


    
\begin{problem} % Problem 2
    Given the following heat problem in a rod of length 1,
    \begin{gather*}
        \mu_t = \mu_{xx}, \qquad 0 < x< 1, \quad t > 0, \\
        \mu(0, t) = 0, \qquad \mu(1, t) + \mu_x(1,t) = 0,\\
        \mu(x, 0) = f(x) = 75.
    \end{gather*}
    \begin{enumerate}[label=\alph*.)]
        \item Find $\mu(x,t)$.
        \item Find $\lim\limits_{t \to \infty} \mu(x, t)$.
        \item Repeat a) and b) but change the right boundary condition to $\mu(1,t) -\mu_x(1,t) = 0$.
        \item You should be finding a significant change in your answers with the new right boundary condition. Please explain. Your explanation should address directly the effect of the sign change on your results.
    \end{enumerate}
\end{problem}

\begin{proof}[\textbf{Solution}] % Problem 2
    \hfill
    \begin{enumerate}[label=\alph*.)]
        \item Let $\mu(x, t) = X(x)T(t)$. Since the initial condition $f$ is non-homogeneous, neither $X$ nor $T$ can be the zero function. Substituting into the given heat equation produces 
        \begin{align*}
            X(x)T'(t) = X''(x)T(t) &\implies \frac{X''(x)}{X(x)} = \frac{T'(t)}{T(t)} = -\lambda &&\implies \left\{\begin{array}{l}
                X''(x) + \lambda X(x) = 0,\\
                T'(t) + \lambda T(t) = 0,
            \end{array}\right.
            \intertext{with boundary conditions}
            \mu(0,t) &\implies X(0)T(t) = 0 &&\implies X(0) = 0, \\
            \mu(1, t) + \mu_x(1, t) &\implies X(1)T(t) + X'(1)T(t) = 0 &&\implies X(1) + X'(1).
        \end{align*}
        
        This is a regular Sturm-Liouville boundary value problem in $X$. 
       
        \underline{If $\lambda = 0$,} then $X = \alpha x + \beta$ and $X' = \alpha$.  \begin{align*}
        \shortintertext{Since $X(0)=0$,}
        \alpha \cdot 0+\beta = 0 \implies &\beta = 0  &\text{so}  &  &X=\alpha x \text{ and } X' = \alpha.
        \shortintertext{Since $X(1) + X'(1) = 0$,}
        1\alpha + \alpha = 0 \implies & \alpha = 0 &\text{so} && X \equiv 0.
        \end{align*}
        Since $X\equiv 0$ cannot be an eigenfunction, \underline{$\lambda = 0$ is not an eigenvalue for the SL-BVP}.
        
        \underline{If $\lambda > 0$,} \begin{equation*}
        X=C_1 \cos \left(\sqrt{\lambda}x\right) + C_2 \sin\left(\sqrt{\lambda}x\right), \qquad
        X'=-C_1 \sqrt{\lambda}\sin \left(\sqrt{\lambda}x\right) + C_2 \sqrt{\lambda}\cos\left(\sqrt{\lambda}x\right)
        \end{equation*}
        \begin{gather*}
        \shortintertext{Since $X(0)=0$,}
        C_1 + 0 = 0 \implies  C_1 = 0  \qquad\text{so} \qquad \begin{cases}
                X=C_2 \sin(\sqrt{\lambda}x), \\ X'=C_2\sqrt{\lambda}\cos(\sqrt{\lambda}x)\end{cases}
        \shortintertext{Since $X(1) + X'(1) = 0$ and $C_2 \neq 0$,}
        C_2\sin(\sqrt{\lambda}) + C_2\sqrt{\lambda}\cos(\sqrt{\lambda}) = 0 \quad \implies\quad \tan(\sqrt{\lambda}) = -\sqrt{\lambda}
        \end{gather*}
        Therefore, the eigenvalues $\lambda_n > 0$ are the solutions to the transcendental equation $\tan(\sqrt{\lambda_n}) = -\sqrt{\lambda_n}$ with corresponding eigenfunctions $X \sim \sin(\sqrt{\lambda_n}x)$.
        
        \underline{If $\lambda < 0$},  let $\lambda = -a^2$ with $a>0$. Then
        \begin{equation*}
        X = C_1 \cosh (a x) + C_2 \sinh (ax) \quad\text{and}\quad X' =  C_1 a\sinh (a x) + C_2 a\cosh (ax).
        \end{equation*}
        \begin{align*}
        \shortintertext{Since $X(0)=0$,}
        &&X(0) = C_1 \cosh (0) + C_2 \sinh (0) &= 0\\
        &\iff& C_1 \cdot 1 + C_2 \cdot 0 &= 0 \\
        &\iff& C_1 &= 0 &\text{so} &  &\begin{cases}
        X=C_2 \sinh(ax), \\ X'=C_2a\cosh(ax)\end{cases}
        \shortintertext{Since $X(1) + X'(1) = 0$ and $C_2 \neq 0$,}
        &&\cancel{C_2} \sinh(a) + \cancel{C_2} a \cosh(a) &= 0\\
        &\iff& \tanh(a) &= -a\\
        \end{align*}
        Since the transcendental equation $\tanh(a) = -a$ has no solutions for $a>0$, there are no eigenvalues $\lambda <0$. 

        The formal solution to the given heat problem can therefore be expressed as
        \begin{equation*}
        \mu(x, t) = \sum_{n=1}^\infty c_n \sin(\sqrt{\lambda_n}x) e^{-\lambda_n t}.
        \end{equation*}
        Since $\mu(x, 0) = f(x) = 75$,
        \begin{align*}
            \mu(x, 0) = \sum_{n=1}^\infty c_n \sin(\sqrt{\lambda_n}x) &= f(x) = 75
            \shortintertext{so}
            c_n = \frac{\displaystyle \int_0^1 f(x) X_n(x)\,\mathrm{d}x}{\displaystyle\int_0^1 X^2_n(x)\,\mathrm{d}x} &= \frac{\displaystyle 75\int_0^1  \sin(\sqrt{\lambda_n}x)\,\mathrm{d}x}{\displaystyle\int_0^1 \sin^2(\sqrt{\lambda_n}x)\,\mathrm{d}x} = \frac{\displaystyle \frac{75}{\sqrt{\lambda_n}}\left.\cos(\sqrt{\lambda_n}x)\right|_0^1}{\displaystyle \int_0^1 \left(\frac{1}{2} - \frac{1}{2} \cos(2\sqrt{\lambda_n}x)\right)\,\mathrm{d}x} \\
            &= \frac{\displaystyle \frac{75}{\sqrt{\lambda_n}}\left(\cos(\sqrt{\lambda_n} )- 1\right)}{\displaystyle \frac{1}{2} - \frac{1}{4\sqrt{\lambda_n}}\left.\sin(2\sqrt{\lambda_n}x)\right|_0^1 } = \frac{\displaystyle \frac{75}{\sqrt{\lambda_n}}\left(\cos(\sqrt{\lambda_n} )- 1\right)}{\displaystyle \frac{1}{2} - \frac{1}{4\sqrt{\lambda_n}}\sin(2\sqrt{\lambda_n})}
        \end{align*}
        The complete solution is therefore
        \begin{equation*}
         \boxed{\mu(x, t) = \sum_{n=1}^\infty \left(\frac{\displaystyle 75\left(\cos(\sqrt{\lambda_n} )- 1\right)}{\displaystyle \frac{\sqrt{\lambda_n}}{2} - \frac{1}{4}\sin(2\sqrt{\lambda_n})} \right)\sin(\sqrt{\lambda_n}x) e^{-\lambda_n t}}.
        \end{equation*}
        where the eigenvalues $\lambda_n$ satisfy the transcendental equation $\tan(\sqrt{\lambda_n}) = -\sqrt{\lambda_n}$.
        
        \item \begin{equation*}
        \lim\limits_{t \to \infty} \mu(x, t) =  \lim\limits_{t \to \infty}\sum_{n=1}^\infty c_n \sin(\sqrt{\lambda_n}x) e^{-\lambda_n t} \geq \sum_{n=1}^\infty \lim\limits_{t \to \infty} c_n \sin(\sqrt{\lambda_n}x) e^{-\lambda_n t} = \sum_{n=1}^\infty 0 = 0.
        \end{equation*}
        
        \item Using the same setup as part a), we produce the regular SL-BVP
        \begin{gather*}
            X''(x) + \lambda X(x) = 0, \qquad 0 < x < 1, \\
            X(0) = 0, \qquad X(1) - X'(1) = 0.
        \end{gather*}
        
        
    \end{enumerate}
\end{proof}

%\begin{problem}
%    In class used the F.S.S of $f(x)=1$, $0 \leq x \leq \pi$,
%    to show that \begin{equation*}
%     1-\frac{1}{3}+\frac{1}{5}-\frac{1}{7}+\cdots = \frac{\pi}{4}.
%     \end{equation*} 
%     Use the F.C.S of $f(x)=x^2$, $0\leq x \leq \pi$, to find \begin{equation*}
%        1 + \frac{1}{2^2} + \frac{1}{3^2} + \frac{1}{4^2} + \cdots
%     \end{equation*}
%\end{problem}
%
%\begin{proof}[\textbf{Solution}]
%    The F.C.S of $f$ is \begin{equation*}
%        f(x) \sim \frac{a_0}2 + \sum_{n=1}^\infty a_n \cos \left(nx\right),
%        \quad 
%        a_0 = \frac{2}{\pi} \int_0^\pi f(x) \, \mathrm dx, \quad
%        a_n = \frac{2}{\pi} \int_0^\pi f(x) \cos \left(nx\right) \, \mathrm dx
%    \end{equation*}
%    Computing the coefficients,
%    \begin{align*}
%        a_0 &= \frac{2}{\pi} \int_0^\pi x^2 \, \mathrm dx = \frac{2}{\pi} \left[ \frac{1}{3}x^3\right]_0^\pi = \frac{2}{3}\pi^2,\\
%        a_n &= \frac{2}{\pi} \int_0^\pi x^2 \cos \left(nx\right) \, \mathrm dx
%            = \frac{2}{\pi} \left[ \cancelto{0}{\left.\frac{x^2}{n}\sin(nx)\right|_0^\pi } - \frac{2}{n} \int_0^\pi x \sin (nx)\,\mathrm dx \right] \\
%            &=\frac{-4}{\pi n} \left[ \left.\frac{-x}{n}\cos(n x) \right|_0^\pi+ \frac{1}{n} \int_0^\pi  \cos (nx)\,\mathrm dx \right]
%            = \frac{-4}{\pi n} \left[ \left(\frac{\pi}{n}(-1)^{n+1}\right) + \cancelto{0}{\left.\left(\frac{1}{n^2}\sin(nx) \right)\right|_0^\pi} \right]\\
%            &=\frac{4}{n^2}(-1)^{n}
%    \end{align*}
%    Therefore, 
%    \begin{equation*}
%        f(x) = x^2 \sim  \frac{\pi^2}{3} + 4\sum_{n=1}^\infty \left(\frac{1}{n^2 }(-1)^n\right)\cos(nx)
%    \end{equation*}
%    
%    Since this series converges to the even piecewise extension of $f$, the series evaluated at the endpoint $x=\pi$ will equal $f(\pi)$. As such, we can compute:
%    \begin{align*}
%    &&f(\pi) = \pi^2 &= \frac{\pi^2}{3} + 4\sum_{n=1}^\infty \left(\frac{1}{n^2 }(-1)^n\right)\cos(n\pi) = \frac{\pi^2}{3} + 4\sum_{n=1}^\infty \frac{1}{n^2 }(-1)^n(-1)^n = \frac{\pi^2}{3} + 4\sum_{n=1}^\infty \frac{1}{n^2 } \\
%    &\iff& \frac{2\pi^2}{3} & = \hphantom{\frac{\pi^2}{3} +\ } 4\sum_{n=1}^\infty \frac{1}{n^2 } = 4\left(1 + \frac{1}{2^2} + \frac{1}{3^2} + \frac{1}{4^2} + \cdots\right) \\
%    &\iff& \frac{1}{4} \cdot \frac{2\pi^2}{3} &= \hphantom{\frac{\pi^2}{3} + 4} \sum_{n=1}^\infty \frac{1}{n^2 } = 1 + \frac{1}{2^2} + \frac{1}{3^2} + \frac{1}{4^2} + \cdots\\
%    &\iff& \boxed{\frac{\pi^2}{6}} &= \hphantom{\frac{\pi^2}{3} + 4} \sum_{n=1}^\infty \frac{1}{n^2 } = 1 + \frac{1}{2^2} + \frac{1}{3^2} + \frac{1}{4^2} + \cdots
%%    &&f(0) = 0&= \frac{\pi}{2} + \sum_{n=1}^\infty \left(\frac{2}{n^2 \pi}\Big[(-1)^n - 1\Big]\right) \cdot 1 \\
%%     &\iff& -\frac{\pi}{2}&=\sum_{n=1}^\infty \left(\frac{2}{n^2 \pi}\Big[(-1)^{n}-1\Big]\right) = \frac{-4}{\pi} + \frac{-4}{3^2\pi} + \frac{-4}{5^2 \pi} + \cdots\\
%%    &\iff& \frac{\pi}{2}&=\sum_{n=1}^\infty \left(\frac{2}{n^2 \pi}\Big[1-(-1)^{n+1}\Big]\right) = \frac{4}{\pi} + \frac{4}{3^2\pi} + \frac{4}{5^2 \pi} + \cdots\\
%%    &\iff& \frac{\pi}{4}\cdot\frac{\pi}{2}&=\sum_{n=1}^\infty \left(\frac{2}{n^2 \pi}\Big[1-(-1)^{n+1}\Big]\right) = \frac{\pi}{4}\left(\frac{4}{\pi} + \frac{4}{3^2\pi} + \frac{4}{5^2 \pi} + \cdots\right)\\ 
%%    &\iff& \boxed{\frac{\pi^2}{8}} &= 1 + \frac{1}{3^2} + \frac{1}{5^2} + \frac{1}{7^2}+\cdots
%    \end{align*}
%    
%\end{proof}
%
%\begin{problem} % 2
%    Consider the boundary value problem
%    \begin{align*}
%        &f'' + \lambda f = 0 &  &0 \leq x \leq \pi \\
%        &f(0) = 0 && f(\pi) -2f'(0) = 0
%    \end{align*}
%    
%    \begin{enumerate}[label=\alph*.)]
%        \item One of the hypotheses of a SL BVP with separated boundary conditions doesn't hold. Which one?
%        \item Is $\lambda =0$ an e-value?
%        \item Are there any e-values for $\lambda < 0$? If so find them.
%        \item Are there any e-values for $\lambda > 0$? If so find them.
%    \end{enumerate}
%\end{problem}
%
%\begin{proof}[\textbf{Solution}] % 2
%\begin{enumerate}[label=\alph*.)]
%    \item For a regular SL-BVP with $x\in (a,b)$, the boundary conditions are defined to be of the form
%    \begin{align*}
%        \kappa_1 f(a) = \kappa_2 f'(a) = 0,    \qquad       \kappa_3 f(b) + \kappa_4 f'(b) = 0
%    \end{align*}
%    In the given problem, the second boundary condition is given in terms of both the left- and right-hand boundaries. However, in the expected form, the second boundary condition should be given in terms of only the right-hand boundary.
%    
%    \item If $\lambda = 0$, then $f = \alpha x + \beta$ and $f' = \alpha$.  \begin{align*}
%    \intertext{ Since $f(0)=0$,}
%        \alpha \cdot 0+\beta = 0 \implies &\beta = 0  &\text{so}  &  &f=\alpha x \text{ and } f' = \alpha
%     \intertext{Since $f(\pi) - 2f'(0) = 0$,}
%        \alpha \pi -2\alpha = \alpha(\pi-2) = 0 \implies &\alpha = 0 &\text{so} &  &f \equiv 0
%    \end{align*}
%    Since $f\equiv 0$ cannot be an eigenfunction, \underline{$\lambda = 0$ is not an eigenvalue for the SL-BVP}.
%    
%    \item  \underline{For $\lambda < 0$}, let $\lambda = -a^2$. Then $f = C_1 \cosh (a x) + C_2 \sinh (ax)$ and \\ \hspace*{2.06in} $f' =  C_1 a\sinh (a x) + C_2 a\cosh (ax)$.
%    
%    Since $f(0)=0$,
%    \begin{align*}
%        &&f(0) = C_1 \cosh (0) + C_2 \sinh (0) &= 0\\
%        &\iff& C_1 \cdot 1 + C_2 \cdot 0 &= 0 \\
%        &\iff& C_1 &= 0 &\text{so} &  &\begin{cases}
%        f=C_2 \sinh(ax), \\ f'=C_2a\cosh(ax)\end{cases}
%    \intertext{Since $f(\pi) - 2f'(0) = 0$,}
%        &&f(\pi) - 2f'(0) = \cancel{C_2} \sinh(a\pi) - 2\cancel{C_2} a\cdot 1 &= 0 \\
%        &\iff& \sinh(a\pi) &= 2a
%    \end{align*}
%    
%    Therefore, the eigenvalues $\lambda < 0$ will be the solutions to the transcendental equation $\sinh (a \pi) = 2a$, $a=\sqrt{-\lambda}$. To find these solutions, we consider the graphs of both sides of the equation, $y_1 = \sinh (a\pi)$ and $y_2 = 2a$. Since at  $a=0$ (outside the domain of consideration), $y_1(0) = y_2(0)=0$, the origin is contained in graphs of both. For $a>0$, we note that \begin{equation*}
%    y'_1 = \pi \cosh (a \pi) > \pi >  y'_2 = 2. 
%    \end{equation*}
%    Since $y_1$ always increases faster than $y_2$ on the interval, there will be no intersections for $a>0$. Therefore, \underline{there are no eignenvalues  $\lambda < 0$}. This result is confirmed graphically by plotting $y_1$ and $y_2$:
%    
%    \begin{center}
%        \begin{tikzpicture}
%        \begin{axis}[
%        axis lines=center,
%        xlabel={$a=\sqrt{-\lambda}$},
%%        xlabel style={below midway},
%        ylabel style={above left},
%        xmin=-0.5,
%        xmax=4,
%        ymin=-1,
%        ymax=10,
%        thick,
%        small,
%        ] 
%        \addplot+[->, mark=none, domain=0:0.93] {sinh(x*pi)} node[pos=0.8, right]
%         {\footnotesize $y_1=\sinh{a\pi}$}; 
%         
%         
%        \addplot+[->, mark=none, domain=0:3] {2*x} node[pos=0.9, below, yshift=-20] {\footnotesize $y_2=2a$};
%        \addplot[mark=o] coordinates {(0,0)};
%        \end{axis}
%        \end{tikzpicture}
%    \end{center}
%    
%    \item \underline{For $\lambda > 0$}, \begin{equation*}
%        f=C_1 \cos \left(\sqrt{\lambda}x\right) + C_2 \sin\left(\sqrt{\lambda}x\right), \qquad
%        f'=-C_1 \sqrt{\lambda}\sin \left(\sqrt{\lambda}x\right) + C_2 \sqrt{\lambda}\cos\left(\sqrt{\lambda}x\right)
%    \end{equation*}
%    \begin{align*}
%    \intertext{Since $f(0)=0$,}
%        C_1 \cdot 1 + C_2 \cdot 0 = 0 \implies &C_1 = 0 &\text{so} &  &\begin{cases}
%        f=C_2 \sin(\sqrt{\lambda}x), \\ f'=C_2\sqrt{\lambda}\cos(\sqrt{\lambda}x)\end{cases}
%    \intertext{Since $f(\pi) - 2f'(0) = 0$,}
%        \cancel{C_2} \sin(\sqrt{\lambda}\pi) - 2\cancel{C_2} \sqrt{\lambda}\cdot 1 = 0 \implies &\sin(\sqrt{\lambda}\pi) = 2\sqrt{\lambda} &&&
%    \end{align*}
%    Therefore, the eigenvalues $\lambda > 0$ will be the solutions to the transcendental equation $\sin (\sqrt{\lambda} \pi) = 2\sqrt{\lambda}$. These solutions can be found graphically by ploting $y_1=\sin (\sqrt{\lambda} \pi)$ and $y_2 =  2\sqrt{\lambda}$:
%    
%    \begin{center}
%        \begin{tikzpicture}
%        \begin{axis}[
%        axis lines=center,
%        xlabel={$x=\sqrt{\lambda}$},
%        %        xlabel style={below midway},
%        ylabel style={above left},
%        xmin=-0.5,
%        xmax={0.62*pi},
%        ymin=-1,
%        ymax=2,
%        thick,
%        small,
%        ] 
%        \addplot+[->, mark=none, domain=0:0.6*pi] {sin(deg(x*pi))} node[pos=0.7, left]
%        {\footnotesize $y_1=\sin{x\pi}$}; 
%        \addplot+[->, mark=none, domain=0:1] {2*x} node[pos=0.7, right] {\footnotesize $y_2=2x$};
%        \addplot[mark=o] coordinates {(0,0)};
%        \addplot[mark=*] coordinates {(0.5,1)} node[above left, xshift=2] {\footnotesize $\left(\frac{1}{2}, 1\right)$};
%        \addplot[mark=none, dashed] coordinates {(0.5, 0)  (0.5, 1)};
%        \end{axis}
%        \end{tikzpicture}
%    \end{center}
%    This plot indicates that the only solution is $\sqrt \lambda = \frac{1}{2}$.
%    
%    Therefore, for $\lambda > 0$, the only eignenvalue is $\lambda_1 = \frac{1}{4}$, with a corresponding eigenfunction of $f_1 \sim \sin \left( \frac{x}{2} \right)$
%    
%\end{enumerate}
%\end{proof}
%
%
%\begin{problem} % 3
%    Consider the boundary value problem
%    \begin{align*}
%    &f'' + \lambda f = 0 &  &0 \leq x \leq \pi \\
%    &f(0) + f'(0) = 0 && f(\pi) -f'(\pi) = 0
%    \end{align*}
%    
%    \begin{enumerate}[label=\alph*.)]
%        \item Determine if there are \underline{negative} e-values, and if so display them graphically and estimate.
%        \item State what happens to the negative e-value(s) of this SL BVP if the right end point is $p$ where $p>\pi$.
%    \end{enumerate}
%\end{problem}
%
%\begin{proof}[\textbf{Solution}] % 3
%    \begin{enumerate}[label=\alph*.)]
%        \item  \underline{For $\lambda < 0$}, let $\lambda = -a^2$, $a>0$. Then \begin{equation*}
%        f = C_1 \cosh (a x) + C_2 \sinh (ax), \qquad f' =  C_1 a\sinh (a x) + C_2 a\cosh (ax).
%        \end{equation*}
%        
%        Since $f(0) + f'(0) = 0$,
%        \begin{align*}
%            &&&f(0) + f'(0) = C_1 \cosh 0 + C_2 \sinh 0 + C_1 a \sinh 0 + C_2 a \cosh 0 = 0\\
%            &\iff& &C_1 + C_2 a = 0 \\
%            &\iff& &C_1 = -C_2 a  \qquad \text{so}\qquad \begin{cases}
%                f=-C_2 a \cosh (ax) + C_2 \sinh(ax),\\
%                f'=-C_2 a^2 \sinh(ax) + C_2 a \cosh(ax)
%            \end{cases}
%        \end{align*}
%        Since $f(\pi) -f'(\pi) = 0$, and $C_2 \neq 0$,
%        \begin{align*}
%            && &f(\pi) - f'(\pi) = \Big(-C_2a \cosh (a\pi) + C_2 \sinh (a\pi)\Big) - \Big(-  C_2a^2\sinh (a \pi) + C_2 a\cosh (a\pi)\Big) =0 \\
%            &\iff&& -a \cosh(a\pi) + \sinh(a\pi) + a^2 \sinh(a\pi) - a \cosh(a\pi) = 0\\
%            &\iff&& (-2a)\cosh(a\pi) + (1+a^2)\sinh(a\pi) = 0\\
%            \intertext{Since $\cosh(a\pi) \neq 0$,},
%            &&&(-2a)\frac{\cosh(a\pi)}{\cosh(a\pi)} + (1+a^2)\frac{\sinh(a\pi)}{\cosh(a\pi)} = 0 \\
%            &\iff& &-2a+(1+a^2)\tanh(a\pi) = 0 \\
%            &\iff&& \tanh (a\pi) = \frac{2a}{1+a^2}
%        \end{align*}
%        
%        Therefore, the eignevalues $\lambda <0$ are the solutions to the transcendental equation $\tanh (a\pi) = 2a/(1+a^2)$, $a=\sqrt{-\lambda}$. These solutions may be found graphically by plotting $y_1 = \tanh (a\pi) - 2a/(1+a^2)$ and determining the intercepts with the $x$-axis:
%        \begin{center}
%            \begin{tikzpicture}
%            \begin{axis}[
%            axis lines=center,
%            xlabel={$a=\sqrt{-\lambda}$},
%            xlabel style={right},
%            ylabel style={above left},
%            xmin=-2.0,
%            xmax=2.0,
%            ymin=-0.3,
%            ymax=0.3,
%            thick,
%            samples=100,
%            clip=false,            
%            ]
%            \addplot+[<-, dashed, mark=none, domain=-1.5:0, blue] {(1+\x^2)*tanh(\x*pi) - 2*\x};
%
%            \addplot+[->, mark=none, domain=0:1.5, blue] {tanh(\x*pi) - (2*\x)/(1+\x^2)} node[above right] {\footnotesize $y_1=\tanh(a\pi)-2a/(1+a^2)$};
%            
%            \addplot[mark=o] coordinates {(0,0)};
%            \addplot[mark=*] coordinates {(0.88225181948493,0) (1.0715068662678,0)};
%            \end{axis} 
%            \end{tikzpicture}
%        \end{center}
%        This yields two solutions, $\sqrt{-\lambda_1}\approx 0.88225$ and $\sqrt{-\lambda_2}\approx 1.07151$.
%    
%    \item As the right end points $p$ increases, the two intercepts found above grow closer to $\sqrt{-\lambda}=a=1$:
%    
%    \begin{center}
%        \begin{tikzpicture}
%        \begin{groupplot}[group style={
%                group name=myplot,
%                vertical sep=2cm,
%                horizontal sep=2cm,
%                group size= 2 by 2,
%            },
%            ymax=.5,
%            height=5cm,
%            width=6.4cm
%            ]
%            
%        \nextgroupplot[title={$p=\pi$}, axis lines=center, samples=100,
%            legend to name=firstplot
%        ]
%         % (Relative) Coordinate at top of the first plot
%        \coordinate (c1) at (rel axis cs:0,1);
%        \addplot+[->, mark=none, domain=0:1.5, blue] {tanh(\x*pi) - (2*\x)/(1+\x^2)};
%        \addplot[mark=o] coordinates {(0,0)};
%        \addplot[mark=*] coordinates {(0.88225181948493,0) (1.0715068662678,0)};
%        \addlegendentry{$\tanh(ap)-\frac{2a}{1+a^2}$}
%        
%        \nextgroupplot[title={$p=1.2\pi$}, axis lines=center, samples=100]
%        % (Relative) Coordinate at top of the second plot
%        \coordinate (c2) at (rel axis cs:1,1);
%        \addplot+[->, mark=none, domain=0:1.5, blue] {(1+\x^2)*tanh(\x*1.2*pi) - 2*\x};
%        \addplot[mark=o] coordinates {(0,0)};
%        \addplot[mark=*] coordinates {(0.9447906,0) (1.040395,0)};
%        
%        \nextgroupplot[title={$p=1.5\pi$}, axis lines=center, samples=100]
%        \addplot+[->, mark=none, domain=0:1.5, blue] {tanh(\x*1.5*pi) - (2*\x)/(1+\x^2)};
%        \addplot[mark=o] coordinates {(0,0)};
%        \addplot[mark=*] coordinates {(0.9804959,0) (1.01674,0)};
%        
%        \nextgroupplot[title={$p=2\pi$}, axis lines=center, samples=100]
%        \addplot+[->, mark=none, domain=0:1.5, blue] {tanh(\x*2*pi) - (2*\x)/(1+\x^2)};
%        \addplot[mark=o] coordinates {(0,0)};
%        \addplot[mark=*] coordinates {(0.996182,0) (1.0036567,0)};
%        
%        \end{groupplot}
%        % Sigle legened solution from
%        % https://tex.stackexchange.com/questions/315224/center-legend-above-or-below-a-groupplot-without-references
%        \coordinate (c3) at ($(c1)!.5!(c2)$);
%        \node[below] at (c3 |- current bounding box.south)
%        {\pgfplotslegendfromname{firstplot}};
%        \end{tikzpicture}
%    \end{center}
%    
%    Therefore, for right end points $p$ large, the eignenvalues $\lambda <0$ approach $-1$.
%    \end{enumerate}
%\end{proof}

\end{document}

